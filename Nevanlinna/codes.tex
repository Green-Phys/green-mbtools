\documentclass[onecolumn,amsmath,notitlepage,aps,prl,10pt,superscriptaddress,floatfix,letterpaper
]{revtex4-2}

\usepackage{listings}
\usepackage{babel}
\usepackage{dejavu}
\usepackage{xcolor}
\usepackage[T1]{fontenc}
\usepackage[utf8]{inputenc}

\usepackage[defaultsans]{droidsans}
\renewcommand{\familydefault}{\sfdefault}

\makeatletter
\long\def\@makecaption#1#2{%
  \vskip\abovecaptionskip
    \small #1: #2\par
  \vskip\belowcaptionskip}%
\makeatother

\definecolor{mygray}{rgb}{0.957,0.957,0.957}
%\definecolor{mygreen}{rgb}{0.149,0.514,0.137}
\definecolor{myred}{rgb}{0.85,0,0}
\definecolor{mygreen}{rgb}{0.298,0.6,0}
\definecolor{myorange}{rgb}{1.0,0.4,0}
\definecolor{myblue}{rgb}{0.11,0.196,0.741}

\lstset{
backgroundcolor=\color{mygray},
basicstyle=\scriptsize\ttfamily\color{black},
commentstyle=\color{mygreen},
frame=single,
numbers=left,
numbersep=5pt,
numberstyle=\tiny\color{gray},
keywordstyle=\color{myblue},
showspaces=false,
showstringspaces=false,
stringstyle=\color{myred},
tabsize=2,
breaklines=true,
}

\begin{document}
\small
\noindent \textit{Program Title:} Nevanlinna\\
\textit{Programming Language:} C++\\
\textit{Dependency:} Eigen3 and GMP libraries\\
\begin{itemize}
\itemsep0em
\item Prepare a file for input parameters. Our example needs the file name of the Matsubara Green's function data (ifile), the number of Matsubara points (imag\_num) and where to output spectral function (ofile).
\item Prepare the Matsubara Green's function data file (in the format of `frequency real\_part imag\_part\textbackslash n' with increasing positive Matsubara frequencies)
\item Change the real grid descretization as needed, including the minimum and maximum frequency, number of discretized points and eta \textit{i.e.} $\eta$ (evaluation axis is $\omega + i\eta$) in Listing \ref{nevanlinna.h} line number 74.
\item Change output from $A(\omega)$ to $G^R=-\mathcal{NG}(\omega+i\eta)$ or else as needed in Listing \ref{nevanlinna.h} line number 95.
\item Change calculation precision in Listing \ref{nevanlinna.cpp} line number 10 as needed. A typical sufficient precision for Schur algorithm is 128.
\item Can output the ultimate $\{a(z), b(z), c(z), d(z)\}$ in Listing \ref{schur.h} line number 92 for convenience of calculating the functional norm during optimization, without the need to rerun this program.
\item This program can also be used to evaluate $A(\omega)$ with an optimized $\theta_{M+1}$. Change the constant 0 $\theta_{M+1}$ in Listing \ref{schur.h} line number 90 to the formula for your $\theta_{M+1}(z)$.
\item Compile the program with gnu c++ compiler and run the executable ./nevanlinna with input redirection.
\end{itemize}

\begin{lstlisting}[caption=compile and run the program]
g++ -o nevanlinna nevanlinna.cpp -I path/to/eigen3 -lgmp -lgmpxx
./nevanlinna < input.txt
\end{lstlisting}

\begin{lstlisting}[caption=input.txt]
ifile imag_num ofile
\end{lstlisting}


\lstinputlisting[language=C++, caption=\lstname, label={nevanlinna.cpp},
directivestyle={\color{black}},
emph={int,nev_complex,nev_real,nev_complex_vector,nev_complex_matrix,nev_complex_matrix_vector},
emphstyle={\color{myorange}}]{nevanlinna.cpp}

\lstinputlisting[language=C++, caption=\lstname, label={schur.h},
directivestyle={\color{black}},
emph={int,nev_complex,nev_real,nev_complex_vector,nev_complex_matrix,nev_complex_matrix_vector},
emphstyle={\color{myorange}}]{schur.h}

\lstinputlisting[language=C++, caption=\lstname, label={nevanlinna.h},
directivestyle={\color{black}},
emph={int,nev_complex,nev_real,nev_complex_vector,nev_complex_matrix,nev_complex_matrix_vector},
emphstyle={\color{myorange}}]{nevanlinna.h}
\end{document}
